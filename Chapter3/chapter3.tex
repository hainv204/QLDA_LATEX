\chapter{Bảng biểu, hình vẽ, phương trình}
\label{Chapter3}
%\section{Bảng biểu, hình vẽ, phương trình}
Việc đánh số bảng biểu, hình vẽ, phương trình phải gắn với số chương; ví dụ hình 3.4 có nghĩa là hình thứ 4 trong Chương 3.
Mọi đồ thị, bảng biểu, hình vẽ lấy từ các nguồn khác phải được trích dẫn đầy đủ. Đầu đề của bảng biểu ghi \textbf{phía trên} bảng, đầu đề của hình vẽ ghi \textbf{phía dưới} hình. 
Khi đề cập đến các bảng biểu và hình vẽ phải nêu rõ số của hình và bảng biểu đó, ví dụ ``... được nêu trong Bảng 4.1'' hoặc ``xem Hình 3.2'' mà không được viết ``… được nêu trong bảng dưới đây'' hoặc ``trong đồ thị của X và Y sau''.

\section{Bảng biểu}
Thông thường, những bảng ngắn và đồ thị phải đi liền với phần nội dung đề cập tới các bảng và đồ thị này ở lần thứ nhất.
Các bảng dài có thể để ở những trang riêng nhưng cũng phải tiếp theo ngay phần nội dung đề cập tới bảng này ở lần đầu tiên.

\begin{table}[ht]
\caption{Ví dụ về bảng biểu 
\label{tab:table1}}
\centering
\begin{tabular}{|c||c|c|c|}
\hline
One & Two & Three & Four\\
\hline
A & B & C & D\\
E & F & G & H\\
\hline
\end{tabular}
\end{table}

Để tạo bảng biểu, tham khảo thêm tại \url{sharelatex.com}.
Bảng~\ref{tab:table1} là một ví dụ về bảng.
Ngoài ra, có một số tool online~\footnote{ http://www.tablesgenerator.com/} có thể được dùng để tạo bảng biểu một cách trực quan.

\section{Hình vẽ}
Hình \ref{fig_1} là một ví dụ về sử dụng gói graphicx để chèn hình ảnh trong latex. Chú ý là $\backslash${\tt{label}} phải xuất hiện SAU (hoặc cùng) $\backslash${\tt{caption}}.
Khi chèn hình, $\backslash${\tt{caption}} nên xuất hiện sau $\backslash${\tt{includegraphics}}.

\begin{figure}[ht]
\centering
\includegraphics[width=1.5in]{images/logo.jpg}
\caption{Ví dụ về hình ảnh.}
\label{fig_1}
\end{figure}


\subsection{Phương trình toán học}
Việc trình bày phương trình toán học trên một dòng đơn hoặc dòng kép tùy ý, tuy nhiên phải thống nhất trong toàn báo cáo.

Ví dụ về thêm công thức sử dụng ''equation''. Để đánh số các phương trình, hãy sử dụng macro $\backslash${\tt{label}} để tạo một định danh cho phương trình. LaTeX sẽ tự động đánh số phương trình cho bạn.

\begin{equation}
\label{deqn_ex1}
x = \sum_{i=0}^{n} 2{i} Q.
\end{equation}




