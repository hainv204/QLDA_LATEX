\chapter{Trình bày báo cáo}
\label{Chapter2}

\section{Quy định chung}
Báo cáo phải được trình bày ngắn gọn, rõ ràng, mạch lạc, sạch sẽ, không được tẩy xóa, có đánh số trang, đánh số bảng biểu, hình vẽ, đồ thị. 

Nội dung báo cáo phải được phân thành các chương. Số thứ tự của các chương, mục được đánh số bằng hệ thống số Ả-rập, không dùng số La mã. Các mục và tiểu mục được đánh số bằng các nhóm hai hoặc ba chữ số, cách nhau một dấu chấm: số thứ nhất chỉ số chương, chỉ số thứ hai chỉ số mục, số thứ ba chỉ số tiểu mục.

Báo cáo trình bày sử dụng khổ giấy với việc canh lề như sau: Lề trên 3,5 cm, lề dưới 3.0 cm, lề trái 3,5 cm, lề phải 2 cm. 

Font chữ dùng trong báo cáo (Times New Roman) với kích cỡ (size) 13pt, sử dụng chế độ dãn dòng (line spacing) chế độ 1.5 lines.

\section{Bố cục của báo cáo}

Một báo cáo thường gồm các chương sau:

\begin{itemize}
\item \textbf{Giới thiệu (Chương 1)}: 

- Giới thiệu tổng quan về đề tài, lý do chọn đề tài, mục đích, đối tượng và phạm vi nghiên cứu.

- Mô tả bài toán mà đề tài giải quyết và những hướng giải quyết hiện có của đề tài.

- Những hạn chế, những vấn đề của những hướng giải quyết trước đây là gì?

- Những vấn đề mà đề tài sẽ giải quyết.

- Các đóng góp chính của đề tài.

\item \textbf{Các công trình liên quan (Chương 2)}: 

- Phân tích đánh giá các đề tài/ nghiên cứu đã có của các tác giả trong và ngoài nước liên quan đến đề tài.

- Từ những vấn đề còn tồn tại, chỉ ra những vấn đề mà đề tài sẽ tập trung, nghiên cứu giải quyết.

\item \textbf{Phương pháp đề xuất (Chương 3)}

- Trình bày cơ sở lý thuyết, lý luận, giả thiết khoa học và phương pháp nghiên cứu đã được sử dụng trong đề tài.

- Nếu đề xuất hướng giải quyết mới, mô hình mới thì cần mô tả chi tiết phương pháp mới đó.

\item \textbf{Kết quả thí nghiệm (Chương 4)}

- Mô tả các kết quả nghiên cứu khoa học hoặc kết quả thực nghiệm. Thông thường cần mô tả môi trường thí nghiệm trước như mô tả các tập dữ liệu để kiểm chứng phương pháp/ mô hình. Miêu tả các độ đo để đánh giá kết quả. Môi trường chạy thí nghiệm. So sánh kết quả thực nghiệm với các nghiên cứu liên quan trước đó, bàn luận và giải thích kết quả.

- Đối với đề tài ứng dụng có kết quả là sản phẩm phần mềm phải có hồ sơ thiết kế, cài đặt,\ldots theo một trong các mô hình đã học (UML,\ldots).

\item \textbf{Kết luận (Chương 5)}:

 - Tóm tắt lại những kết quả đạt được, những đóng góp mới và những đề xuất mới, \ldots.
 
 - Những hạn chế mà đề tài chưa giải quyết được (nếu có) và hướng phát triển trong tương lai.

\item \textbf{Tài liệu tham khảo}: 

- Chỉ bao gồm các tài liệu được trích dẫn, sử dụng và đề cập tới để bàn luận trong báo cáo.

- Hướng dẫn trích dẫn ở chương \ref{Chapter4}.

\item \textbf{Phụ lục (nếu có)}:

- Phần này bao gồm nội dung cần thiết nhằm minh họa hoặc hỗ trợ cho nội dung báo cáo như số liệu, mẫu biểu, tranh ảnh,\ldots 

- Nếu có công trình công bố thì sẽ để vào phần phụ lục này.
\end{itemize}

